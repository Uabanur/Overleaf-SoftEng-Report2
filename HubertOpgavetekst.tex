intro
\begin{mdframed}[backgroundcolor=black!5]
Der skal afleveres jeres endlige version af kildekoden, testene, det kørende program og en instruktion hvordan man kører programmet som ZIP fil som indholder et Eclipse projekt og som kan importeres i Eclipse
\end{mdframed}

white box
\begin{mdframed}[backgroundcolor=black!5]
Dette afsnit skal indeholde planer for systematiske
white box tests til udvalgte metoder. \\

Der skal vælges 2 metoder for tomandsgrupper, 3 for tremandsgrupper og 4 for firmandsgrupper. Prøve at vælge metoder som gør noget. Det skal ikke vælges getter og setter metoder. Vis metoderne i rapporten. \\

Benyt samme skabelon til systematiske tests som er brugt i forbindelse med forelæsninger (dvs. to tabeller; den ene med input egenskaber og den anden med konkrete værdier). \\

De planlagte systematiske test skal implementeres og udføres ved brug af JUnit og skal afleveres som en del af Eclipse projektet.
\end{mdframed}

\begin{mdframed}[backgroundcolor=black!5]
 Der skal gives en kort status og vurdering af det afleverede produktet.
Desuden skal der være et kort afsnit, der reflekterer omkring projektets forløb, og
sammenligner det aktuelle programdesign efter implementeringen med programdesign
fra rapport 1.
 \end{mdframed}
 
 \begin{mdframed}[backgroundcolor=black!5]
I dette afsnit skal der redegøres af de brugte design mønstre i jeres design/kode. Hvilke design mønstre har I brugt og hvorfor?  \\
 
 Hvis I har ikke brugt nogen
design mønstre så skal der skrives hvorfor I synes det giver ikke mening at bruge design mønstre. Bemærk, der er ingen krav til at bruge design mønstre i jeres design og kode.\\

Design mønstre er mulige løsninger til nogle særlige design problemer. Hvis I har igen af disse design problemer, eller der findes bedre løsninger, så skal der heller ikke bruges
design mønstre
\end{mdframed}